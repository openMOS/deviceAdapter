\section{Project Demonstrators}
\begin{table*}
	\begin{tabular}{p{10cm}|c|c|c|c|c}
		\hline
		& \textbf{Introsys} & \textbf{Masmec} & \textbf{Loughborough} & \textbf{Ford} & \textbf{Electrolux} \\
		\hline
		\textbf{Plug-and-produce –- (un)plugging a workstation –- MSB adjustment} & \checkmark & \checkmark & (\checkmark) &  & (\checkmark) \\
		\hline
		\textbf{Plug-and-produce –- (un)plugging a module –- DA adjustment}& \checkmark &  &  &  &  \\
		\hline
		\textbf{Monitoring of execution data} & \checkmark & \checkmark & \checkmark & \checkmark & \checkmark \\
		\hline
		\textbf{Semantic description of all elements} & \checkmark & \checkmark & \checkmark & \checkmark & \checkmark \\
		\hline
		\textbf{DA standalone operation} & \checkmark & \checkmark & \checkmark &  & \checkmark \\
		\hline
		\textbf{DA orchestration of composite skills} & \checkmark &  & \checkmark &  & \checkmark \\
		\hline
		\textbf{Performance benchmarking with existing systems} & \checkmark & \checkmark &  &  &  \\
		\hline
		\textbf{KPI visualization} & \checkmark & \checkmark & \checkmark &\checkmark  & \checkmark \\
		\hline
		\textbf{Creation of new recipes for atomic skills} & \checkmark & \checkmark & \checkmark &  & \checkmark \\
		\hline
		\textbf{Creation of new recipes for composite skills} & \checkmark &  & \checkmark &  & \checkmark \\
		\hline
		\textbf{Edit recipes} & \checkmark & \checkmark & \checkmark &  & \checkmark \\
		\hline
		\textbf{Edit execution tables} & \checkmark & \checkmark & \checkmark &  & \checkmark \\
		\hline
		\textbf{System reconfiguration –- Switching a module from one workstation to another} & \checkmark &  &  &  &  \\
		\hline
		\textbf{Runtime production adjustments} & & \checkmark &  &  &  \\
		\hline
		\textbf{Queuing skill executions} &  & &  &  & \checkmark \\
		\hline
		\textbf{Scalability of system } &  &  & \checkmark &  &  \\
		\hline
		\textbf{System optimization based on time and energy} &  &  & \checkmark &  &  \\
		\hline
		\textbf{Support for system ramp-up process } &  &  & \checkmark &  &  \\
		\hline
		\textbf{A system constisting of virtual and physical stations running together} &  & \checkmark & \checkmark &  & \checkmark \\
		\hline
		\textbf{System optimization during ramp-up for energy efficiency} & \checkmark & \checkmark & \checkmark & \checkmark & \checkmark \\
		\hline
		\textbf{Passive mode} &  &  &  & \checkmark &  \\
		\hline
		\textbf{Changing between production phases} & \checkmark & \checkmark & \checkmark &  & \checkmark \\
		\hline
		\textbf{Direct execution of Skills/Recipes/Execution table line} & \checkmark & \checkmark & \checkmark &  & \checkmark \\
		\hline
		\textbf{System rapid adjustment to disruption} & \checkmark & \checkmark & \checkmark &  &  \\
		\hline
		\textbf{Prevent data loss} & \checkmark & \checkmark & \checkmark & \checkmark & \checkmark \\
		\hline
	\end{tabular}
	\\
	\caption{openMOS demonstrators and their key functionalities.}
	\label{tab:demos}
\end{table*}

The following demonstrators were successfully run during the project to show and evaluate the \gls{openMOS} technology:

\begin{itemize}
	\item \textbf{Introsys demonstrator} consists of two robotic spot welding cells that represent actual production lines from the automotive industry following Ford and Volkswagen standards, respectively. 
	The main actor of each cell is a 6-DoF robotic arm, equipped with a custom gripper designed to handle a body part of a car (a longeron). 
	Motion control and safety routines are managed by an industrial field PLC. 
	An additional vision-based quality inspection system allows to identify potential imperfections on the weld. 
	Finally, an energy monitoring module collects and registers the energy consumed by all equipment.
	The \textbf{Introsys demonstrator} presents a \textbf{dynamical (un)plugging a module} and \textbf{system reconfiguration}: a vision system, one shared between two demonstrators, can be dynamically unplugged from one station and attached to another.
	The system will automatically discover the change and rearrange its skills as well as possible execution scenarios.
	\item \textbf{Masmec} utilizes the \textbf{IDEAS demonstrator} that consists of a conveyor with a number of different stations. 
	Some processes can be executed in parallel, which allows to show optimization scenarios. 
	The demonstrator also features the \textbf{runtime production adjustments} functionality - dynamic product routing that depends on the process outcome.
	In Masmec case a product is treated differently based on the leak test outcome: if it a workpiece fails a test the system adjusts to execute a recovery route for this specific workpiece.
	This includes on-the-fly decisions done by the system based on the previous production steps. 
	\item \textbf{Loughborough University demonstrator} shows  the \textbf{scalability} of the \gls{openMOS} solution. 
	It hosts the \textbf{physical and emulated facilities} controlled by \glspl{DA} that are connected via the \gls{MSB} to the \gls{HMI} and the agent cloud. 
	The simulated environment allows to demonstrate how the \gls{openMOS} scales up, running in a distributed control system and test most of the \gls{openMOS} key features.
	\item \textbf{Ford demonstrator} is based on the current sealant dispensing machine used in	the Ford Dunton Technical Centre Prototype build area for the majority of liquid sealant applications.
	The Robotic RTV/Anaerobic Dispensing Cell is a special machine designed to robotically apply precise beads of “room temperature vulcanizing” (RTV) sealant and anaerobic sealant to automotive parts as an aid to the assembly process.
	The key \gls{openMOS} feature that was demonstrated in the Ford case was the \textbf{\gls{openMOS} passive mode} - the ability of the \gls{openMOS} software stack to gather the data from the low level equipment without actively triggering the production.
	\item \textbf{Electrolux} selected an injection moulding cell as a 
	demonstration scenario for the project purposes. 
	The cell produces of plastic components, the front and rear tub shells, which constitute the envelope of a washing group.
	A need to handle a large amount of different product variants, rigid configuration as well as the variety and the age of the hardware and software components constituting such cells leads to quite frequent breakdowns and subsequently human intervention, reducing the cell effectiveness.
	Thus, \textbf{rapid changeover} becomes one of the \gls{openMOS} key functionalities demonstrated here.
	\item \textbf{VDMA OPC UA Demonstrator} is an external demonstrator, not directly related to the \gls{openMOS} project. It was successfully demonstrated how the whole \gls{openMOS} stack can be deployed to the system that was not initially designed to be \gls{openMOS}-compatible.
\end{itemize}
Table~\ref{tab:demos} summarizes the \gls{openMOS} technologies showed in each of the project demonstrators.

